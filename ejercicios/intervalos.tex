
        \documentclass[spanish, 11pt]{exam}

        %These tell TeX which packages to use.
        \usepackage{array,epsfig}
        \usepackage{amsmath, textcomp}
        \usepackage{amsfonts}
        \usepackage{amssymb}
        \usepackage{amsxtra}
        \usepackage{amsthm}
        \usepackage{mathrsfs}
        \usepackage{color}
        \usepackage{multicol, xparse}
        \usepackage{verbatim}


        \usepackage[utf8]{inputenc}
        \usepackage[spanish]{babel}
        \usepackage{eurosym}

        \usepackage{graphicx}
        \graphicspath{{../img/}}
        \usepackage{pgf}
        
        \usepackage{pgfplots}
        \usetikzlibrary{math}
        \pgfplotsset{compat=1.15}
        \usepackage{xfp}

        \printanswers
        \nopointsinmargin
        \pointformat{}

        %Pagination stuff.
        %\setlength{\topmargin}{-.3 in}
        %\setlength{\oddsidemargin}{0in}
        %\setlength{\evensidemargin}{0in}
        %\setlength{\textheight}{9.in}
        %\setlength{\textwidth}{6.5in}
        %\pagestyle{empty}

        \let\multicolmulticols\multicols
        \let\endmulticolmulticols\endmulticols
        \RenewDocumentEnvironment{multicols}{mO{}}
         {%
          \ifnum#1=1
            #2%
          \else % More than 1 column
            \multicolmulticols{#1}[#2]
          \fi
         }
         {%
          \ifnum#1=1
          \else % More than 1 column
            \endmulticolmulticols
          \fi
         }
        \renewcommand{\solutiontitle}{\noindent\textbf{Sol:}\enspace}

        \newcommand{\samedir}{\mathbin{\!/\mkern-5mu/\!}}

        \newcommand{\class}{2º Bachillerato CCSS}
        \newcommand{\examdate}{\today}

        \newcommand{\tipo}{A}


        \newcommand{\timelimit}{50 minutos}



        \pagestyle{head}
        \firstpageheader{\includegraphics[width=0.2\columnwidth]{header_left}}{\textbf{Departamento de Matemáticas\linebreak \class}\linebreak \examnum}{\includegraphics[width=0.1\columnwidth]{header_right}}
        \runningheader{\class}{\examnum}{Página \thepage\ of \numpages}
        \runningheadrule

        \newcommand{\examnum}{Intervalos de confianza}
        \begin{document}
        \begin{questions}
        \question p71e1-Una muestra aleatoria de 36 personas, empleadas en una gran industria, da el número medio de días al año
que faltan al trabajo es $\overline{x} = 12$ con $s = 4$.
a) Dar una estimación puntual de $\mu$.
b) Tomando un nivel del 99\% dar el intervalo de confianza de $\mu$.
c) Tomando un nivel del 95\% dar el intervalo de confianza de $\mu$. \begin{solution}   a) $\mu\approx 12$ \\ b) $\alpha=1-0.99=0.01\to \frac{\alpha}{2}=0.005$ \\ \\ Valor crítico: \\ $P(Z>z_{\alpha/2})=0.005\to P(Z<z_{\alpha/2})=0.995 \to z_{\alpha/2} =2.5758$ \\ 
  \begin{tikzpicture}[scale=0.8]
    \pgfmathdeclarefunction{gauss}{2}{\pgfmathparse{1/(#2*sqrt(2*pi))*exp(-((x-#1)^2)/(2*#2^2))}}
    \tikzmath{\conf = 0.99; \crit= 2.5758; \a=0.005);}
    \begin{axis}[no markers, domain=-5:5, samples=100, axis lines=left, height=5cm, width=12cm, xtick={0,\crit}, ytick=\empty, xticklabels = {$0$, $z_{\frac{\alpha}{2}}=\crit$},enlargelimits=false, clip=false, axis on top]
      \addplot [fill=cyan!20, draw=none, domain=-\crit:\crit] {gauss(0,1)} \closedcycle;
      \addplot [very thick,cyan!50!black] {gauss(0,1)};
    \end{axis}
    \node[] at (5.2,1.5) {$\conf$};	
    \draw[-]   (\crit+6.5,1)node[right]{$\a$}  --  (\crit+5.6,0.1) ;
  \end{tikzpicture} \\
  Error cometido: \\ $E=z_{\alpha/2}\cdot \frac{\sigma}{\sqrt{n}} \to E=2.5758\cdot \frac{4}{6.0}=1.7172$ \\ Por tanto el intervalo de confianza será: \\$\left(\overline{x} - E , \overline{x} + E \right)=\left(12 - 1.7172 , 12 + 1.7172 \right)=\left(10.2828, 13.7172 \right)$ \\  \\ 
  \begin{tikzpicture}[scale=0.4]
      \tikzmath{\a = -10; \b = 10; \aa = \a -1; \bb = \b + 1 ; \dist = \b - \a; \med = (\a + \b)/2;}
      \draw[very thick] (\a,0) -- (\b,0);
      \path [draw=black, fill=white] (\b,0) circle (2pt);
      \path [draw=black, fill=white] (\a,0.0) circle (2pt);
      \draw[latex-latex] (\a - 1.5,0) -- (\b + 1.5,0) ;
      \draw[shift={(\a,0)},color=black] (0pt,3pt) -- (0pt,-3pt);
      \draw[shift={(\a,0)},color=black] (0pt,0pt) -- (0pt,-3pt) node[below] {$10.2828$};
      \draw[shift={(\med,0)},color=black] (0pt,3pt) -- (0pt,-3pt);
      \draw[shift={(\med,0)},color=black] (0pt,0pt) -- (0pt,-3pt) node[below] {$12$};
      \draw[shift={(\b,0)},color=black] (0pt,3pt) -- (0pt,-3pt);
      \draw[shift={(\b,0)},color=black] (0pt,0pt) -- (0pt,-3pt) node[below] {$13.7172$};
      \draw[decorate,decoration={brace}, thick](\med,0.2)--(\b,0.2) node[above, midway] {$E=1.7172$}; 
  \end{tikzpicture} \\
  \\ c) $\alpha=1-0.95=0.05\to \frac{\alpha}{2}=0.025$ \\ \\ Valor crítico: \\ $P(Z>z_{\alpha/2})=0.025\to P(Z<z_{\alpha/2})=0.975 \to z_{\alpha/2} =1.96$ \\ 
  \begin{tikzpicture}[scale=0.8]
    \pgfmathdeclarefunction{gauss}{2}{\pgfmathparse{1/(#2*sqrt(2*pi))*exp(-((x-#1)^2)/(2*#2^2))}}
    \tikzmath{\conf = 0.95; \crit= 1.96; \a=0.025);}
    \begin{axis}[no markers, domain=-5:5, samples=100, axis lines=left, height=5cm, width=12cm, xtick={0,\crit}, ytick=\empty, xticklabels = {$0$, $z_{\frac{\alpha}{2}}=\crit$},enlargelimits=false, clip=false, axis on top]
      \addplot [fill=cyan!20, draw=none, domain=-\crit:\crit] {gauss(0,1)} \closedcycle;
      \addplot [very thick,cyan!50!black] {gauss(0,1)};
    \end{axis}
    \node[] at (5.2,1.5) {$\conf$};	
    \draw[-]   (\crit+6.5,1)node[right]{$\a$}  --  (\crit+5.6,0.1) ;
  \end{tikzpicture} \\
  Error cometido: \\ $E=z_{\alpha/2}\cdot \frac{\sigma}{\sqrt{n}} \to E=1.96\cdot \frac{4}{6.0}=1.3067$ \\ Por tanto el intervalo de confianza será: \\$\left(\overline{x} - E , \overline{x} + E \right)=\left(12 - 1.3067 , 12 + 1.3067 \right)=\left(10.6933, 13.3067 \right)$ \\  \\ 
  \begin{tikzpicture}[scale=0.4]
      \tikzmath{\a = -10; \b = 10; \aa = \a -1; \bb = \b + 1 ; \dist = \b - \a; \med = (\a + \b)/2;}
      \draw[very thick] (\a,0) -- (\b,0);
      \path [draw=black, fill=white] (\b,0) circle (2pt);
      \path [draw=black, fill=white] (\a,0.0) circle (2pt);
      \draw[latex-latex] (\a - 1.5,0) -- (\b + 1.5,0) ;
      \draw[shift={(\a,0)},color=black] (0pt,3pt) -- (0pt,-3pt);
      \draw[shift={(\a,0)},color=black] (0pt,0pt) -- (0pt,-3pt) node[below] {$10.6933$};
      \draw[shift={(\med,0)},color=black] (0pt,3pt) -- (0pt,-3pt);
      \draw[shift={(\med,0)},color=black] (0pt,0pt) -- (0pt,-3pt) node[below] {$12$};
      \draw[shift={(\b,0)},color=black] (0pt,3pt) -- (0pt,-3pt);
      \draw[shift={(\b,0)},color=black] (0pt,0pt) -- (0pt,-3pt) node[below] {$13.3067$};
      \draw[decorate,decoration={brace}, thick](\med,0.2)--(\b,0.2) node[above, midway] {$E=1.3067$}; 
  \end{tikzpicture} \\
     \end{solution}\question p71e2- Cuál debe ser el tamaño de una muestra si se quiere estimar el gasto medio de una familia en electricidad,
mensualmente, con un error menor de 1.500 pesetas y el 95/% como nivel de confianza, sabiendo que el gasto
medio en 1999 era de 13.300 ptas. con una desviación de 1.200 ptas. \begin{solution}   $\alpha=1-0.95=0.05\to \frac{\alpha}{2}=0.025$ \\ \\ Valor crítico: \\ $P(Z>z_{\alpha/2})=0.025\to P(Z<z_{\alpha/2})=0.975 \to z_{\alpha/2} =1.96$ \\ 
  \begin{tikzpicture}[scale=0.8]
    \pgfmathdeclarefunction{gauss}{2}{\pgfmathparse{1/(#2*sqrt(2*pi))*exp(-((x-#1)^2)/(2*#2^2))}}
    \tikzmath{\conf = 0.95; \crit= 1.96; \a=0.025);}
    \begin{axis}[no markers, domain=-5:5, samples=100, axis lines=left, height=5cm, width=12cm, xtick={0,\crit}, ytick=\empty, xticklabels = {$0$, $z_{\frac{\alpha}{2}}=\crit$},enlargelimits=false, clip=false, axis on top]
      \addplot [fill=cyan!20, draw=none, domain=-\crit:\crit] {gauss(0,1)} \closedcycle;
      \addplot [very thick,cyan!50!black] {gauss(0,1)};
    \end{axis}
    \node[] at (5.2,1.5) {$\conf$};	
    \draw[-]   (\crit+6.5,1)node[right]{$\a$}  --  (\crit+5.6,0.1) ;
  \end{tikzpicture} \\
  Error cometido: \\ $E=z_{\alpha/2}\cdot \frac{\sigma}{\sqrt{n}} \to n =\frac{\sigma^2 \cdot z_{\alpha / 2}^2}{E^2} \to n \geq3$ \\    \end{solution}\question p71e3- Una muestra aleatoria de 100 bombillas producidas por una determinada fábrica, tiene una vida media de
1.280 horas con una desviación típica de 140 horas:
i) Estimar la duración media de las bombillas fabricas por esa fábrica.
ii) Dar el intervalo de confianza, a un nivel del 0,95 de la estimación anterior. \begin{solution}   i) $\mu\approx 1280$ \\ ii) $\alpha=1-0.95=0.05\to \frac{\alpha}{2}=0.025$ \\ \\ Valor crítico: \\ $P(Z>z_{\alpha/2})=0.025\to P(Z<z_{\alpha/2})=0.975 \to z_{\alpha/2} =1.96$ \\ 
  \begin{tikzpicture}[scale=0.8]
    \pgfmathdeclarefunction{gauss}{2}{\pgfmathparse{1/(#2*sqrt(2*pi))*exp(-((x-#1)^2)/(2*#2^2))}}
    \tikzmath{\conf = 0.95; \crit= 1.96; \a=0.025);}
    \begin{axis}[no markers, domain=-5:5, samples=100, axis lines=left, height=5cm, width=12cm, xtick={0,\crit}, ytick=\empty, xticklabels = {$0$, $z_{\frac{\alpha}{2}}=\crit$},enlargelimits=false, clip=false, axis on top]
      \addplot [fill=cyan!20, draw=none, domain=-\crit:\crit] {gauss(0,1)} \closedcycle;
      \addplot [very thick,cyan!50!black] {gauss(0,1)};
    \end{axis}
    \node[] at (5.2,1.5) {$\conf$};	
    \draw[-]   (\crit+6.5,1)node[right]{$\a$}  --  (\crit+5.6,0.1) ;
  \end{tikzpicture} \\
  Error cometido: \\ $E=z_{\alpha/2}\cdot \frac{\sigma}{\sqrt{n}} \to E=1.96\cdot \frac{140}{10.0}=27.44$ \\ Por tanto el intervalo de confianza será: \\$\left(\overline{x} - E , \overline{x} + E \right)=\left(1280 - 27.44 , 1280 + 27.44 \right)=\left(1252.56, 1307.44 \right)$ \\  \\ 
  \begin{tikzpicture}[scale=0.4]
      \tikzmath{\a = -10; \b = 10; \aa = \a -1; \bb = \b + 1 ; \dist = \b - \a; \med = (\a + \b)/2;}
      \draw[very thick] (\a,0) -- (\b,0);
      \path [draw=black, fill=white] (\b,0) circle (2pt);
      \path [draw=black, fill=white] (\a,0.0) circle (2pt);
      \draw[latex-latex] (\a - 1.5,0) -- (\b + 1.5,0) ;
      \draw[shift={(\a,0)},color=black] (0pt,3pt) -- (0pt,-3pt);
      \draw[shift={(\a,0)},color=black] (0pt,0pt) -- (0pt,-3pt) node[below] {$1252.56$};
      \draw[shift={(\med,0)},color=black] (0pt,3pt) -- (0pt,-3pt);
      \draw[shift={(\med,0)},color=black] (0pt,0pt) -- (0pt,-3pt) node[below] {$1280$};
      \draw[shift={(\b,0)},color=black] (0pt,3pt) -- (0pt,-3pt);
      \draw[shift={(\b,0)},color=black] (0pt,0pt) -- (0pt,-3pt) node[below] {$1307.44$};
      \draw[decorate,decoration={brace}, thick](\med,0.2)--(\b,0.2) node[above, midway] {$E=27.44$}; 
  \end{tikzpicture} \\
     \end{solution}\question p71e4- La dirección de un hotel quiere saber el número de días que, de promedio, sus clientes se hospedan en él .
Toma una muestra de 400 clientes y obtiene una media de 5,4 días con una desviación típica de 2 días.
i) Estimar el número medio de días que los clientes permanecen en el hotel.
ii) Hallar el intervalo de confianza de la estimación anterior a un nivel del 90\% \begin{solution}   i) $\mu\approx 5.4$ \\ ii) $\alpha=1-0.9=0.1\to \frac{\alpha}{2}=0.05$ \\ \\ Valor crítico: \\ $P(Z>z_{\alpha/2})=0.05\to P(Z<z_{\alpha/2})=0.95 \to z_{\alpha/2} =1.6449$ \\ 
  \begin{tikzpicture}[scale=0.8]
    \pgfmathdeclarefunction{gauss}{2}{\pgfmathparse{1/(#2*sqrt(2*pi))*exp(-((x-#1)^2)/(2*#2^2))}}
    \tikzmath{\conf = 0.9; \crit= 1.6449; \a=0.05);}
    \begin{axis}[no markers, domain=-5:5, samples=100, axis lines=left, height=5cm, width=12cm, xtick={0,\crit}, ytick=\empty, xticklabels = {$0$, $z_{\frac{\alpha}{2}}=\crit$},enlargelimits=false, clip=false, axis on top]
      \addplot [fill=cyan!20, draw=none, domain=-\crit:\crit] {gauss(0,1)} \closedcycle;
      \addplot [very thick,cyan!50!black] {gauss(0,1)};
    \end{axis}
    \node[] at (5.2,1.5) {$\conf$};	
    \draw[-]   (\crit+6.5,1)node[right]{$\a$}  --  (\crit+5.6,0.1) ;
  \end{tikzpicture} \\
  Error cometido: \\ $E=z_{\alpha/2}\cdot \frac{\sigma}{\sqrt{n}} \to E=1.6449\cdot \frac{2}{20.0}=0.1645$ \\ Por tanto el intervalo de confianza será: \\$\left(\overline{x} - E , \overline{x} + E \right)=\left(5.4 - 0.1645 , 5.4 + 0.1645 \right)=\left(5.2355, 5.5645 \right)$ \\  \\ 
  \begin{tikzpicture}[scale=0.4]
      \tikzmath{\a = -10; \b = 10; \aa = \a -1; \bb = \b + 1 ; \dist = \b - \a; \med = (\a + \b)/2;}
      \draw[very thick] (\a,0) -- (\b,0);
      \path [draw=black, fill=white] (\b,0) circle (2pt);
      \path [draw=black, fill=white] (\a,0.0) circle (2pt);
      \draw[latex-latex] (\a - 1.5,0) -- (\b + 1.5,0) ;
      \draw[shift={(\a,0)},color=black] (0pt,3pt) -- (0pt,-3pt);
      \draw[shift={(\a,0)},color=black] (0pt,0pt) -- (0pt,-3pt) node[below] {$5.2355$};
      \draw[shift={(\med,0)},color=black] (0pt,3pt) -- (0pt,-3pt);
      \draw[shift={(\med,0)},color=black] (0pt,0pt) -- (0pt,-3pt) node[below] {$5.4$};
      \draw[shift={(\b,0)},color=black] (0pt,3pt) -- (0pt,-3pt);
      \draw[shift={(\b,0)},color=black] (0pt,0pt) -- (0pt,-3pt) node[below] {$5.564500000000001$};
      \draw[decorate,decoration={brace}, thick](\med,0.2)--(\b,0.2) node[above, midway] {$E=0.1645$}; 
  \end{tikzpicture} \\
     \end{solution}\question p71e5- Una muestra aleatoria de 225 votantes tiene 135 partidarios para el pago de impuestos destinados a la mejora
de la ciudad.
i) Efectuar una estimación puntual del porcentaje de votantes que están a favor de pagar el nuevo
impuesto.
ii) Dar el intervalo de confianza, a un nivel del 0,95 de la estimación anterior. \begin{solution}   i) $p \approx \widehat{p} =60$\% \\ ii) $\alpha=1-0.95=0.05\to \frac{\alpha}{2}=0.025$ \\ \\ Valor crítico: \\ $P(Z>z_{\alpha/2})=0.025\to P(Z<z_{\alpha/2})=0.975 \to z_{\alpha/2} =1.96$ \\ 
  \begin{tikzpicture}[scale=0.8]
    \pgfmathdeclarefunction{gauss}{2}{\pgfmathparse{1/(#2*sqrt(2*pi))*exp(-((x-#1)^2)/(2*#2^2))}}
    \tikzmath{\conf = 0.95; \crit= 1.96; \a=0.025);}
    \begin{axis}[no markers, domain=-5:5, samples=100, axis lines=left, height=5cm, width=12cm, xtick={0,\crit}, ytick=\empty, xticklabels = {$0$, $z_{\frac{\alpha}{2}}=\crit$},enlargelimits=false, clip=false, axis on top]
      \addplot [fill=cyan!20, draw=none, domain=-\crit:\crit] {gauss(0,1)} \closedcycle;
      \addplot [very thick,cyan!50!black] {gauss(0,1)};
    \end{axis}
    \node[] at (5.2,1.5) {$\conf$};	
    \draw[-]   (\crit+6.5,1)node[right]{$\a$}  --  (\crit+5.6,0.1) ;
  \end{tikzpicture} \\
  Error cometido: \\ $E=z_{\alpha / 2}\cdot \sqrt{\frac{\widehat{p}\cdot\left(1-\widehat{p} \right)}{n}} \to E=1.96\cdot \sqrt{\frac{0.6\cdot0.4}{225}}=0.064$ \\ Por tanto el intervalo de confianza será: \\$\left(\widehat{p} - E , \widehat{p} + E \right)=\left(0.6 - 0.064 , 0.6 + 0.064 \right)=\left(0.536, 0.664 \right)$ \\  \\ 
  \begin{tikzpicture}[scale=0.4]
      \tikzmath{\a = -10; \b = 10; \aa = \a -1; \bb = \b + 1 ; \dist = \b - \a; \med = (\a + \b)/2;}
      \draw[very thick] (\a,0) -- (\b,0);
      \path [draw=black, fill=white] (\b,0) circle (2pt);
      \path [draw=black, fill=white] (\a,0.0) circle (2pt);
      \draw[latex-latex] (\a - 1.5,0) -- (\b + 1.5,0) ;
      \draw[shift={(\a,0)},color=black] (0pt,3pt) -- (0pt,-3pt);
      \draw[shift={(\a,0)},color=black] (0pt,0pt) -- (0pt,-3pt) node[below] {$0.536$};
      \draw[shift={(\med,0)},color=black] (0pt,3pt) -- (0pt,-3pt);
      \draw[shift={(\med,0)},color=black] (0pt,0pt) -- (0pt,-3pt) node[below] {$0.6$};
      \draw[shift={(\b,0)},color=black] (0pt,3pt) -- (0pt,-3pt);
      \draw[shift={(\b,0)},color=black] (0pt,0pt) -- (0pt,-3pt) node[below] {$0.6639999999999999$};
      \draw[decorate,decoration={brace}, thick](\med,0.2)--(\b,0.2) node[above, midway] {$E=0.064$}; 
  \end{tikzpicture} \\
     \end{solution}\question p71e6- El valor de diez cheques al descubierto ha sido de 670, 1400, 350, 1700, 2000, 1700, 530, 710, 1200 y 1000
euros. Si estas diez denuncias se consideran como una muestra del total de todas las que por ese motivo se
realizan a lo largo de un año, ¿cuál será el valor medio de los cheques al descubierto que se expedirán. \begin{solution}   Calculamos la media estadística para estimar la media: $\overline{x}=\frac{{\sum_{i=1}^n x_i }}{n} =1126.0$ \\   \end{solution}\question p71e7- De 120 alumnos, la proporción de que tengan dos o más hermanos es de 48/120. Indica los parámetros de la
distribución a las que se ajustarían las muestras de tamaño 30. \begin{solution}   La distribución muestral $\widehat{p} \rightarrow N \left ( p , \sqrt{ \frac{p \cdot (1-p)} {n}}\right )$. Luego: \\ $p \approx \widehat{p}=\frac{48}{120}=\frac{2}{5}$ \\ $\sigma \approx \sqrt{\frac{\frac{48}{120}\cdot \left(1-\frac{48}{120}\right)}{30}}=\frac{\sqrt{5}}{25}$ \\    \end{solution}\question p71e8- En cierto instituto de Enseñanza Secundaria hay matriculados 800 alumnos. A una muestra seleccionada
aleatoriamente de un 15\% de ellos, se les preguntó si utilizaban la cafetería del centro. Contestaron
negativamente un total de 24 alumnos. Estima el porcentaje de alumnado que utiliza la cafetería del instituto.
Determina, con un confianza del 99\%, el error máximo cometido con dicha estimación \begin{solution}   i) Tamaño de la muestra:$120.0$. Alumnos de la muestra que utiliza la cafetería: $96.0\to $ \\ $p \approx \widehat{p} =80$\% \\ ii) $\alpha=1-0.95=0.05\to \frac{\alpha}{2}=0.025$ \\ \\ Valor crítico: \\ $P(Z>z_{\alpha/2})=0.025\to P(Z<z_{\alpha/2})=0.975 \to z_{\alpha/2} =1.96$ \\ 
  \begin{tikzpicture}[scale=0.8]
    \pgfmathdeclarefunction{gauss}{2}{\pgfmathparse{1/(#2*sqrt(2*pi))*exp(-((x-#1)^2)/(2*#2^2))}}
    \tikzmath{\conf = 0.95; \crit= 1.96; \a=0.025);}
    \begin{axis}[no markers, domain=-5:5, samples=100, axis lines=left, height=5cm, width=12cm, xtick={0,\crit}, ytick=\empty, xticklabels = {$0$, $z_{\frac{\alpha}{2}}=\crit$},enlargelimits=false, clip=false, axis on top]
      \addplot [fill=cyan!20, draw=none, domain=-\crit:\crit] {gauss(0,1)} \closedcycle;
      \addplot [very thick,cyan!50!black] {gauss(0,1)};
    \end{axis}
    \node[] at (5.2,1.5) {$\conf$};	
    \draw[-]   (\crit+6.5,1)node[right]{$\a$}  --  (\crit+5.6,0.1) ;
  \end{tikzpicture} \\
  Error cometido: \\ $E=z_{\alpha / 2}\cdot \sqrt{\frac{\widehat{p}\cdot\left(1-\widehat{p} \right)}{n}} \to E=1.96\cdot \sqrt{\frac{0.8\cdot0.2}{120.0}}=0.0716$ \\ Por tanto el intervalo de confianza será: \\$\left(\widehat{p} - E , \widehat{p} + E \right)=\left(0.8 - 0.0716 , 0.8 + 0.0716 \right)=\left(0.7284, 0.8716 \right)$ \\  \\ 
  \begin{tikzpicture}[scale=0.4]
      \tikzmath{\a = -10; \b = 10; \aa = \a -1; \bb = \b + 1 ; \dist = \b - \a; \med = (\a + \b)/2;}
      \draw[very thick] (\a,0) -- (\b,0);
      \path [draw=black, fill=white] (\b,0) circle (2pt);
      \path [draw=black, fill=white] (\a,0.0) circle (2pt);
      \draw[latex-latex] (\a - 1.5,0) -- (\b + 1.5,0) ;
      \draw[shift={(\a,0)},color=black] (0pt,3pt) -- (0pt,-3pt);
      \draw[shift={(\a,0)},color=black] (0pt,0pt) -- (0pt,-3pt) node[below] {$0.7284$};
      \draw[shift={(\med,0)},color=black] (0pt,3pt) -- (0pt,-3pt);
      \draw[shift={(\med,0)},color=black] (0pt,0pt) -- (0pt,-3pt) node[below] {$0.8$};
      \draw[shift={(\b,0)},color=black] (0pt,3pt) -- (0pt,-3pt);
      \draw[shift={(\b,0)},color=black] (0pt,0pt) -- (0pt,-3pt) node[below] {$0.8716$};
      \draw[decorate,decoration={brace}, thick](\med,0.2)--(\b,0.2) node[above, midway] {$E=0.0716$}; 
  \end{tikzpicture} \\
  \\ Por tanto, el error máximo es: 0.0716   \end{solution}\question p71e9- Para estimar la proporción de las familias de una determinada ciudad que poseen microondas, 
se quiere
utilizar una muestra aleatoria de medida n. Calcula el valor mínimo de n para garantizar que, a un nivel de
confianza del 95\%, el error en la estimación sea menor que 0,005. (Como se desconoce la proporción, se ha de
tomar el caso más desfavorable que será 0,5) \begin{solution}   $\alpha=1-0.95=0.05\to \frac{\alpha}{2}=0.025$ \\ \\ Valor crítico: \\ $P(Z>z_{\alpha/2})=0.025\to P(Z<z_{\alpha/2})=0.975 \to z_{\alpha/2} =1.96$ \\ 
  \begin{tikzpicture}[scale=0.8]
    \pgfmathdeclarefunction{gauss}{2}{\pgfmathparse{1/(#2*sqrt(2*pi))*exp(-((x-#1)^2)/(2*#2^2))}}
    \tikzmath{\conf = 0.95; \crit= 1.96; \a=0.025);}
    \begin{axis}[no markers, domain=-5:5, samples=100, axis lines=left, height=5cm, width=12cm, xtick={0,\crit}, ytick=\empty, xticklabels = {$0$, $z_{\frac{\alpha}{2}}=\crit$},enlargelimits=false, clip=false, axis on top]
      \addplot [fill=cyan!20, draw=none, domain=-\crit:\crit] {gauss(0,1)} \closedcycle;
      \addplot [very thick,cyan!50!black] {gauss(0,1)};
    \end{axis}
    \node[] at (5.2,1.5) {$\conf$};	
    \draw[-]   (\crit+6.5,1)node[right]{$\a$}  --  (\crit+5.6,0.1) ;
  \end{tikzpicture} \\
  Error cometido: \\ $E=z_{\alpha / 2}\cdot \sqrt{\frac{\widehat{p}\cdot\left(1-\widehat{p} \right)}{n}} \to n =\frac{\widehat{p}\cdot\left(1-\widehat{p} \right)\cdot z_{\alpha / 2}^2}{E^2} \to n \geq38416$ \\    \end{solution}
    \end{questions}
    \end{document}
    