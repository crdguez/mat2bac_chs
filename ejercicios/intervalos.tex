
        \documentclass[spanish, 11pt]{exam}

        %These tell TeX which packages to use.
        \usepackage{array,epsfig}
        \usepackage{amsmath, textcomp}
        \usepackage{amsfonts}
        \usepackage{amssymb}
        \usepackage{amsxtra}
        \usepackage{amsthm}
        \usepackage{mathrsfs}
        \usepackage{color}
        \usepackage{multicol, xparse}
        \usepackage{verbatim}


        \usepackage[utf8]{inputenc}
        \usepackage[spanish]{babel}
        \usepackage{eurosym}

        \usepackage{graphicx}
        \graphicspath{{../img/}}
        \usepackage{pgf}
        
        \usepackage{pgfplots}
        \usetikzlibrary{math}
        \pgfplotsset{compat=1.15}
        \usepackage{xfp}

        \printanswers
        \nopointsinmargin
        \pointformat{}

        %Pagination stuff.
        %\setlength{\topmargin}{-.3 in}
        %\setlength{\oddsidemargin}{0in}
        %\setlength{\evensidemargin}{0in}
        %\setlength{\textheight}{9.in}
        %\setlength{\textwidth}{6.5in}
        %\pagestyle{empty}

        \let\multicolmulticols\multicols
        \let\endmulticolmulticols\endmulticols
        \RenewDocumentEnvironment{multicols}{mO{}}
         {%
          \ifnum#1=1
            #2%
          \else % More than 1 column
            \multicolmulticols{#1}[#2]
          \fi
         }
         {%
          \ifnum#1=1
          \else % More than 1 column
            \endmulticolmulticols
          \fi
         }
        \renewcommand{\solutiontitle}{\noindent\textbf{Sol:}\enspace}

        \newcommand{\samedir}{\mathbin{\!/\mkern-5mu/\!}}

        \newcommand{\class}{2º Bachillerato CCSS}
        \newcommand{\examdate}{\today}

        \newcommand{\tipo}{A}


        \newcommand{\timelimit}{50 minutos}



        \pagestyle{head}
        \firstpageheader{\includegraphics[width=0.2\columnwidth]{header_left}}{\textbf{Departamento de Matemáticas\linebreak \class}\linebreak \examnum}{\includegraphics[width=0.1\columnwidth]{header_right}}
        \runningheader{\class}{\examnum}{Página \thepage\ of \numpages}
        \runningheadrule

        \newcommand{\examnum}{Intervalos de confianza}
        \begin{document}
        \begin{questions}
        \question Intervalo de confianza para la media, si la media muestral es: 200.8, la desviación típica: 15, tamaño de la muestra: 25 y el grado de confianza: 90.0\%. \\ \\  \begin{solution}   $\alpha=1-0.9=0.1\to \frac{\alpha}{2}=0.05$ \\ \\ Valor crítico: \\ $P(Z>z_{\alpha/2})=0.05\to P(Z<z_{\alpha/2})=0.95 \to z_{\alpha/2} =1.6449$ \\ 
  \begin{tikzpicture}[scale=0.8]
    \pgfmathdeclarefunction{gauss}{2}{\pgfmathparse{1/(#2*sqrt(2*pi))*exp(-((x-#1)^2)/(2*#2^2))}}
    \tikzmath{\conf = 0.9; \crit= 1.6449; \a=0.05);}
    \begin{axis}[no markers, domain=-5:5, samples=100, axis lines=left, height=5cm, width=12cm, xtick={0,\crit}, ytick=\empty, xticklabels = {$0$, $z_{\frac{\alpha}{2}}=\crit$},enlargelimits=false, clip=false, axis on top]
      \addplot [fill=cyan!20, draw=none, domain=-\crit:\crit] {gauss(0,1)} \closedcycle;
      \addplot [very thick,cyan!50!black] {gauss(0,1)};
    \end{axis}
    \node[] at (5.2,1.5) {$\conf$};	
    \draw[-]   (\crit+6.5,1)node[right]{$\a$}  --  (\crit+5.6,0.1) ;
  \end{tikzpicture} \\
  Error cometido: \\ $E=z_{\alpha/2}\cdot \frac{\sigma}{\sqrt{n}} \to E=1.6449\cdot \frac{15}{5.0}=4.9347$ \\ Por tanto el intervalo de confianza será: \\$\left(\overline{x} - E , \overline{x} + E \right)=\left(200.8 - 4.9347 , 200.8 + 4.9347 \right)=\left(195.8653, 205.7347 \right)$ \\  \\ 
  \begin{tikzpicture}[scale=0.4]
      \tikzmath{\a = -10; \b = 10; \aa = \a -1; \bb = \b + 1 ; \dist = \b - \a; \med = (\a + \b)/2;}
      \draw[very thick] (\a,0) -- (\b,0);
      \path [draw=black, fill=white] (\b,0) circle (2pt);
      \path [draw=black, fill=white] (\a,0.0) circle (2pt);
      \draw[latex-latex] (\a - 1.5,0) -- (\b + 1.5,0) ;
      \draw[shift={(\a,0)},color=black] (0pt,3pt) -- (0pt,-3pt);
      \draw[shift={(\a,0)},color=black] (0pt,0pt) -- (0pt,-3pt) node[below] {$195.86530000000002$};
      \draw[shift={(\med,0)},color=black] (0pt,3pt) -- (0pt,-3pt);
      \draw[shift={(\med,0)},color=black] (0pt,0pt) -- (0pt,-3pt) node[below] {$200.8$};
      \draw[shift={(\b,0)},color=black] (0pt,3pt) -- (0pt,-3pt);
      \draw[shift={(\b,0)},color=black] (0pt,0pt) -- (0pt,-3pt) node[below] {$205.7347$};
      \draw[decorate,decoration={brace}, thick](\med,0.2)--(\b,0.2) node[above, midway] {$E=4.9347$}; 
  \end{tikzpicture} \\
     \end{solution}\question Intervalo de confianza para la media, si la media muestral es: 1053, la desviación típica: 75, tamaño de la muestra: 150 y el grado de confianza: 98.0\%. \\ \\  \begin{solution}   $\alpha=1-0.98=0.02\to \frac{\alpha}{2}=0.01$ \\ \\ Valor crítico: \\ $P(Z>z_{\alpha/2})=0.01\to P(Z<z_{\alpha/2})=0.99 \to z_{\alpha/2} =2.3263$ \\ 
  \begin{tikzpicture}[scale=0.8]
    \pgfmathdeclarefunction{gauss}{2}{\pgfmathparse{1/(#2*sqrt(2*pi))*exp(-((x-#1)^2)/(2*#2^2))}}
    \tikzmath{\conf = 0.98; \crit= 2.3263; \a=0.01);}
    \begin{axis}[no markers, domain=-5:5, samples=100, axis lines=left, height=5cm, width=12cm, xtick={0,\crit}, ytick=\empty, xticklabels = {$0$, $z_{\frac{\alpha}{2}}=\crit$},enlargelimits=false, clip=false, axis on top]
      \addplot [fill=cyan!20, draw=none, domain=-\crit:\crit] {gauss(0,1)} \closedcycle;
      \addplot [very thick,cyan!50!black] {gauss(0,1)};
    \end{axis}
    \node[] at (5.2,1.5) {$\conf$};	
    \draw[-]   (\crit+6.5,1)node[right]{$\a$}  --  (\crit+5.6,0.1) ;
  \end{tikzpicture} \\
  Error cometido: \\ $E=z_{\alpha/2}\cdot \frac{\sigma}{\sqrt{n}} \to E=2.3263\cdot \frac{75}{12.24744871391589}=14.2456$ \\ Por tanto el intervalo de confianza será: \\$\left(\overline{x} - E , \overline{x} + E \right)=\left(1053 - 14.2456 , 1053 + 14.2456 \right)=\left(1038.7544, 1067.2456 \right)$ \\  \\ 
  \begin{tikzpicture}[scale=0.4]
      \tikzmath{\a = -10; \b = 10; \aa = \a -1; \bb = \b + 1 ; \dist = \b - \a; \med = (\a + \b)/2;}
      \draw[very thick] (\a,0) -- (\b,0);
      \path [draw=black, fill=white] (\b,0) circle (2pt);
      \path [draw=black, fill=white] (\a,0.0) circle (2pt);
      \draw[latex-latex] (\a - 1.5,0) -- (\b + 1.5,0) ;
      \draw[shift={(\a,0)},color=black] (0pt,3pt) -- (0pt,-3pt);
      \draw[shift={(\a,0)},color=black] (0pt,0pt) -- (0pt,-3pt) node[below] {$1038.7544$};
      \draw[shift={(\med,0)},color=black] (0pt,3pt) -- (0pt,-3pt);
      \draw[shift={(\med,0)},color=black] (0pt,0pt) -- (0pt,-3pt) node[below] {$1053$};
      \draw[shift={(\b,0)},color=black] (0pt,3pt) -- (0pt,-3pt);
      \draw[shift={(\b,0)},color=black] (0pt,0pt) -- (0pt,-3pt) node[below] {$1067.2456$};
      \draw[decorate,decoration={brace}, thick](\med,0.2)--(\b,0.2) node[above, midway] {$E=14.2456$}; 
  \end{tikzpicture} \\
     \end{solution}
    \end{questions}
    \end{document}
    