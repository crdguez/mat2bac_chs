
        \documentclass[spanish, 11pt]{exam}

        %These tell TeX which packages to use.
        \usepackage{array,epsfig}
        \usepackage{amsmath, textcomp}
        \usepackage{amsfonts}
        \usepackage{amssymb}
        \usepackage{amsxtra}
        \usepackage{amsthm}
        \usepackage{mathrsfs}
        \usepackage{color}
        \usepackage{multicol, xparse}
        \usepackage{verbatim}


        \usepackage[utf8]{inputenc}
        \usepackage[spanish]{babel}
        \usepackage{eurosym}

        \usepackage{graphicx}
        \graphicspath{{../img/}}
        \usepackage{pgf}
        
        \usepackage{pgfplots}
        \usetikzlibrary{math}
        \pgfplotsset{compat=1.15}
        \usepackage{xfp}

        \printanswers
        \nopointsinmargin
        \pointformat{}

        %Pagination stuff.
        %\setlength{\topmargin}{-.3 in}
        %\setlength{\oddsidemargin}{0in}
        %\setlength{\evensidemargin}{0in}
        %\setlength{\textheight}{9.in}
        %\setlength{\textwidth}{6.5in}
        %\pagestyle{empty}

        \let\multicolmulticols\multicols
        \let\endmulticolmulticols\endmulticols
        \RenewDocumentEnvironment{multicols}{mO{}}
         {%
          \ifnum#1=1
            #2%
          \else % More than 1 column
            \multicolmulticols{#1}[#2]
          \fi
         }
         {%
          \ifnum#1=1
          \else % More than 1 column
            \endmulticolmulticols
          \fi
         }
        \renewcommand{\solutiontitle}{\noindent\textbf{Sol:}\enspace}

        \newcommand{\samedir}{\mathbin{\!/\mkern-5mu/\!}}

        \newcommand{\class}{2º Bachillerato CCSS}
        \newcommand{\examdate}{\today}

        \newcommand{\tipo}{A}


        \newcommand{\timelimit}{50 minutos}



        \pagestyle{head}
        \firstpageheader{\includegraphics[width=0.2\columnwidth]{header_left}}{\textbf{Departamento de Matemáticas\linebreak \class}\linebreak \examnum}{\includegraphics[width=0.1\columnwidth]{header_right}}
        \runningheader{\class}{\examnum}{Página \thepage\ of \numpages}
        \runningheadrule

        \newcommand{\examnum}{Ampliación Estadística}
        \begin{document}
        \begin{questions}
        \question Para estimar el peso medio de las chicas de
16 años de una ciudad, se toma una mues-
tra aleatoria de 100 de ellas. Se obtienen
los siguientes parámetros: $\overline{x}$ = 52,5 kg,
s = 5,3 kg.
Se realiza la afirmación siguiente:
“El peso medio de las chicas de 16 años de
esta ciudad está entre 51 kg y 54 kg”.
¿Con qué nivel de confianza se hace la
afirmación? \begin{solution}   Calculamos el grado de confianza para un error máximo: 1.5, siendo el tamaño muestral: 100 y la desviación típica: 5.3. \\ \\  \\ Como $E=z_{\alpha / 2}\cdot \frac{\sigma}{\sqrt{n}} \Rightarrow z_{\alpha / 2} =\frac{E \cdot \sqrt{n}}{\sigma}\to z_{\alpha / 2}=2.83$ \\ $P(Z>z_{\alpha / 2})=P(Z>2.83)=\frac{\alpha}{2}=0.002327$ \\ $\alpha=0.004655 \to \quad confianza=1 - \alpha=0.9953 \to 99.53 \%$    \end{solution}\question Un fabricante de lámparas de bajo consumo sabe que el tiempo de duración, en horas, de las
lámparas que fabrica sigue una distribución normal de media desconocida y desviación típica 180
horas. Con una muestra de dichas lámparas elegida al azar y con un nivel de confianza del 97\%,
obtuvo para la media el intervalo de confianza (10 072,1; 10 127,9).
a) Calcula el valor que obtuvo para la media de la muestra y el tamaño de muestra utilizado.
b) Si se quiere que el error de su estimación sea como máximo de 24 horas y se utiliza una muestra de
tamaño 225, ¿cuál será entonces el nivel de confianza? \begin{solution}   a) \\ Del intervalo obtenemos la media muestral: 10100.0 y el error máximo: $10100.0-10072.1=27.9$ \\ Calculamos el tamaño de la muestra para estimar la media con un error máximo: 27.9, si la desviación típica es: 180 y el grado de confianza: 97.0\%. \\ \\ Valor crítico: \\ $\alpha=1-0.97=0.03\to \frac{\alpha}{2}=0.015$ \\ \\ $P(Z>z_{\alpha/2})=0.015\to P(Z<z_{\alpha/2})=0.985 \to z_{\alpha/2} =2.1701$ \\ 
    \begin{tikzpicture}[scale=0.8]
    \pgfmathdeclarefunction{gauss}{2}{\pgfmathparse{1/(#2*sqrt(2*pi))*exp(-((x-#1)^2)/(2*#2^2))}}
    \tikzmath{\conf = 0.97; \crit= 2.1701; \a=0.015);}
    \begin{axis}[no markers, domain=-5:5, samples=100, axis lines=left, height=5cm, width=12cm, xtick={0,\crit}, ytick=\empty, xticklabels = {$0$, $z_{\frac{\alpha}{2}}=\crit$},enlargelimits=false, clip=false, axis on top]
      \addplot [fill=cyan!20, draw=none, domain=-\crit:\crit] {gauss(0,1)} \closedcycle;
      \addplot [very thick,cyan!50!black] {gauss(0,1)};
    \end{axis}
    \node[] at (5.2,1.5) {$\conf$};	
    \draw[-]   (\crit+6.5,1)node[right]{$\a$}  --  (\crit+5.6,0.1) ;
    \end{tikzpicture} \\
    Error cometido: \\ $E=z_{\alpha/2}\cdot \frac{\sigma}{\sqrt{n}} \to n \approx \frac{\sigma^2 \cdot z_{\alpha / 2}^2}{E^2}=196.0180649323622 \to n \geq197$ \\ b) \\ Calculamos el grado de confianza para un error máximo: 24, siendo el tamaño muestral: 225 y la desviación típica: 180. \\ \\  \\ Como $E=z_{\alpha / 2}\cdot \frac{\sigma}{\sqrt{n}} \Rightarrow z_{\alpha / 2} =\frac{E \cdot \sqrt{n}}{\sigma}\to z_{\alpha / 2}=2.0$ \\ $P(Z>z_{\alpha / 2})=P(Z>2.0)=\frac{\alpha}{2}=0.02275$ \\ $\alpha=0.0455 \to \quad confianza=1 - \alpha=0.9545 \to 95.45 \%$    \end{solution}\question EVAU - Junio 2011. La edad a la que obtienen el permiso de conducir los habitantes de una determinada población
es una variable aleatoria que se puede aproximar por una distribución normal de media 24 años y
desviación típica 4 años. Se elige aleatoriamente una muestra de 100 habitantes de dicha población. Sea $\overline{X}$
la media muestral de la edad de obtención del permiso de conducir.
a) ¿Cuáles son la media y la varianza de $\overline{X}$ ?

b) Halle el intervalo de confianza al 90\% para $\overline{X}$ .
 \begin{solution}   a) \\ La distribución $\overline{X} \approx N\left(\mu,\frac{\sigma}{\sqrt{n}}\right)$. Por tanto: \\ $\mu_{\overline{X}}=mu=24$ y $\sigma_{\overline{X}}=\frac{\sigma}{\sqrt{n}}=0.4 \to var({\overline{X}})=\sigma_{\overline{X}}^2=0.16$ \\ b) \\ Calculamos el intervalo de confianza para la media, sabiendo que la media muestral es: 24, la desviación típica: 4, tamaño de la muestra: 100 y el grado de confianza: 90.0\%. \\ \\ Valor crítico: \\ $\alpha=1-0.9=0.1\to \frac{\alpha}{2}=0.05$ \\ \\ $P(Z>z_{\alpha/2})=0.05\to P(Z<z_{\alpha/2})=0.95 \to z_{\alpha/2} =1.6449$ \\ 
    \begin{tikzpicture}[scale=0.8]
    \pgfmathdeclarefunction{gauss}{2}{\pgfmathparse{1/(#2*sqrt(2*pi))*exp(-((x-#1)^2)/(2*#2^2))}}
    \tikzmath{\conf = 0.9; \crit= 1.6449; \a=0.05);}
    \begin{axis}[no markers, domain=-5:5, samples=100, axis lines=left, height=5cm, width=12cm, xtick={0,\crit}, ytick=\empty, xticklabels = {$0$, $z_{\frac{\alpha}{2}}=\crit$},enlargelimits=false, clip=false, axis on top]
      \addplot [fill=cyan!20, draw=none, domain=-\crit:\crit] {gauss(0,1)} \closedcycle;
      \addplot [very thick,cyan!50!black] {gauss(0,1)};
    \end{axis}
    \node[] at (5.2,1.5) {$\conf$};	
    \draw[-]   (\crit+6.5,1)node[right]{$\a$}  --  (\crit+5.6,0.1) ;
    \end{tikzpicture} \\
    Error cometido: \\ $E=z_{\alpha/2}\cdot \frac{\sigma}{\sqrt{n}} \to E=1.6449\cdot \frac{4}{10.0}=0.658$ \\ Por tanto el intervalo de confianza será: \\$\left(\overline{x} - E , \overline{x} + E \right)=\left(24 - 0.658 , 24 + 0.658 \right)=\left(23.342, 24.658 \right)$ \\  \\ 
    \begin{tikzpicture}[scale=0.4]
      \tikzmath{\a = -10; \b = 10; \aa = \a -1; \bb = \b + 1 ; \dist = \b - \a; \med = (\a + \b)/2;}
      \draw[very thick] (\a,0) -- (\b,0);
      \path [draw=black, fill=white] (\b,0) circle (2pt);
      \path [draw=black, fill=white] (\a,0.0) circle (2pt);
      \draw[latex-latex] (\a - 1.5,0) -- (\b + 1.5,0) ;
      \draw[shift={(\a,0)},color=black] (0pt,3pt) -- (0pt,-3pt);
      \draw[shift={(\a,0)},color=black] (0pt,0pt) -- (0pt,-3pt) node[below] {$23.342$};
      \draw[shift={(\med,0)},color=black] (0pt,3pt) -- (0pt,-3pt);
      \draw[shift={(\med,0)},color=black] (0pt,0pt) -- (0pt,-3pt) node[below] {$24$};
      \draw[shift={(\b,0)},color=black] (0pt,3pt) -- (0pt,-3pt);
      \draw[shift={(\b,0)},color=black] (0pt,0pt) -- (0pt,-3pt) node[below] {$24.658$};
      \draw[decorate,decoration={brace}, thick](\med,0.2)--(\b,0.2) node[above, midway] {$E=0.658$}; 
    \end{tikzpicture} \\
       \end{solution}\question EVAU - Septiembre 2015. La producción en kilos de los naranjos de una variedad es una variable
aleatoria con distribución normal de desviación típica igual a 5 kilos.
a) Queremos construir un intervalo de confianza al 96\% para la media de la producción de los
naranjos de esta variedad de forma que su amplitud no sea mayor que 3 kilos. ¿Qué tamaño de la muestra
debemos
tomar? b).Decidimos tomar un tamaño de la muestra igual a 10. Elegimos 10 naranjos de esta
variedad y medimos su producción en kilos, con los siguientes resultados:
82 , 90 , 87 , 75 , 78 , 83 , 92 , 77 , 85 , 86
Calcular el intervalo de confianza al 96\% para la media de la producción de los naranjos de esta variedad. \begin{solution}   a) \\Calculamos el tamaño de la muestra para estimar la media con un error máximo: 1.5, si la desviación típica es: 5 y el grado de confianza: 96.0\%. \\ \\ Valor crítico: \\ $\alpha=1-0.96=0.04\to \frac{\alpha}{2}=0.02$ \\ \\ $P(Z>z_{\alpha/2})=0.02\to P(Z<z_{\alpha/2})=0.98 \to z_{\alpha/2} =2.0537$ \\ 
    \begin{tikzpicture}[scale=0.8]
    \pgfmathdeclarefunction{gauss}{2}{\pgfmathparse{1/(#2*sqrt(2*pi))*exp(-((x-#1)^2)/(2*#2^2))}}
    \tikzmath{\conf = 0.96; \crit= 2.0537; \a=0.02);}
    \begin{axis}[no markers, domain=-5:5, samples=100, axis lines=left, height=5cm, width=12cm, xtick={0,\crit}, ytick=\empty, xticklabels = {$0$, $z_{\frac{\alpha}{2}}=\crit$},enlargelimits=false, clip=false, axis on top]
      \addplot [fill=cyan!20, draw=none, domain=-\crit:\crit] {gauss(0,1)} \closedcycle;
      \addplot [very thick,cyan!50!black] {gauss(0,1)};
    \end{axis}
    \node[] at (5.2,1.5) {$\conf$};	
    \draw[-]   (\crit+6.5,1)node[right]{$\a$}  --  (\crit+5.6,0.1) ;
    \end{tikzpicture} \\
    Error cometido: \\ $E=z_{\alpha/2}\cdot \frac{\sigma}{\sqrt{n}} \to n \approx \frac{\sigma^2 \cdot z_{\alpha / 2}^2}{E^2}=46.863152111111106 \to n \geq47$ \\ b) \\ La media muestral es: 83.5 \\Calculamos el intervalo de confianza para la media, sabiendo que la media muestral es: 83.5, la desviación típica: 5, tamaño de la muestra: 10 y el grado de confianza: 96.0\%. \\ \\ Valor crítico: \\ $\alpha=1-0.96=0.04\to \frac{\alpha}{2}=0.02$ \\ \\ $P(Z>z_{\alpha/2})=0.02\to P(Z<z_{\alpha/2})=0.98 \to z_{\alpha/2} =2.0537$ \\ 
    \begin{tikzpicture}[scale=0.8]
    \pgfmathdeclarefunction{gauss}{2}{\pgfmathparse{1/(#2*sqrt(2*pi))*exp(-((x-#1)^2)/(2*#2^2))}}
    \tikzmath{\conf = 0.96; \crit= 2.0537; \a=0.02);}
    \begin{axis}[no markers, domain=-5:5, samples=100, axis lines=left, height=5cm, width=12cm, xtick={0,\crit}, ytick=\empty, xticklabels = {$0$, $z_{\frac{\alpha}{2}}=\crit$},enlargelimits=false, clip=false, axis on top]
      \addplot [fill=cyan!20, draw=none, domain=-\crit:\crit] {gauss(0,1)} \closedcycle;
      \addplot [very thick,cyan!50!black] {gauss(0,1)};
    \end{axis}
    \node[] at (5.2,1.5) {$\conf$};	
    \draw[-]   (\crit+6.5,1)node[right]{$\a$}  --  (\crit+5.6,0.1) ;
    \end{tikzpicture} \\
    Error cometido: \\ $E=z_{\alpha/2}\cdot \frac{\sigma}{\sqrt{n}} \to E=2.0537\cdot \frac{5}{3.1622776601683795}=3.2472$ \\ Por tanto el intervalo de confianza será: \\$\left(\overline{x} - E , \overline{x} + E \right)=\left(83.5 - 3.2472 , 83.5 + 3.2472 \right)=\left(80.2528, 86.7472 \right)$ \\  \\ 
    \begin{tikzpicture}[scale=0.4]
      \tikzmath{\a = -10; \b = 10; \aa = \a -1; \bb = \b + 1 ; \dist = \b - \a; \med = (\a + \b)/2;}
      \draw[very thick] (\a,0) -- (\b,0);
      \path [draw=black, fill=white] (\b,0) circle (2pt);
      \path [draw=black, fill=white] (\a,0.0) circle (2pt);
      \draw[latex-latex] (\a - 1.5,0) -- (\b + 1.5,0) ;
      \draw[shift={(\a,0)},color=black] (0pt,3pt) -- (0pt,-3pt);
      \draw[shift={(\a,0)},color=black] (0pt,0pt) -- (0pt,-3pt) node[below] {$80.2528$};
      \draw[shift={(\med,0)},color=black] (0pt,3pt) -- (0pt,-3pt);
      \draw[shift={(\med,0)},color=black] (0pt,0pt) -- (0pt,-3pt) node[below] {$83.5$};
      \draw[shift={(\b,0)},color=black] (0pt,3pt) -- (0pt,-3pt);
      \draw[shift={(\b,0)},color=black] (0pt,0pt) -- (0pt,-3pt) node[below] {$86.7472$};
      \draw[decorate,decoration={brace}, thick](\med,0.2)--(\b,0.2) node[above, midway] {$E=3.2472$}; 
    \end{tikzpicture} \\
       \end{solution}\question EVAU - Junio 2012. La cantidad de horas que duermen los vecinos de un pueblo de Zaragoza se puede aproximar
por una distribución normal con una desviación típica de 0,64. Se toma una muestra aleatoria simple y se
obtienen los siguientes datos (en horas que duermen cada noche):
6,9 7,3 7,6 6,6 6,5 7,1 6,2 6,9 7,8 6,7 7,0 6,5 5,5 7,2 7,6 5,8
a) Calcular la media muestral del número de horas que se duerme cada noche.
b) Determinar el nivel de confianza para el cual el intervalo de confianza para la media de
horas que se duerme cada noche es (6,65 , 7). Detallar los pasos realizados para obtener los resultados. \begin{solution}   a) \\ La media muestral es: 6.825 \\b) \\ Del intervalo obtenemos la media muestral: 6.825 y el error máximo: $6.825-6.65=0.175$ \\ Calculamos el grado de confianza para un error máximo: 0.175, siendo el tamaño muestral: 16 y la desviación típica: 0.64. \\ \\  \\ Como $E=z_{\alpha / 2}\cdot \frac{\sigma}{\sqrt{n}} \Rightarrow z_{\alpha / 2} =\frac{E \cdot \sqrt{n}}{\sigma}\to z_{\alpha / 2}=1.09$ \\ $P(Z>z_{\alpha / 2})=P(Z>1.09)=\frac{\alpha}{2}=0.1379$ \\ $\alpha=0.2757 \to \quad confianza=1 - \alpha=0.7243 \to 72.43 \%$    \end{solution}
    \end{questions}
    \end{document}
    